
\section{Exercise 5}
Figure \ref{fig:loktavolterra} shows the numerical solution of the Lokta-Volterra prey-predator model by the 4 stage Runge-Kutta method with parameters $\alpha = 0.2$, $\beta = 0.01$, $\gamma = 0.004$, $\delta = 0.07$, step size $h=10^{-3}$, final time $T=300$ and initial condition $x(0)= 19$, $y(0)=22$.
\begin{figure}[ht!]
    \centering
\includegraphics[width=0.4\linewidth]{Pic/loktavolterra.jpg}
    \caption{Prey Predator trajectory in time.}
    \label{fig:loktavolterra}
\end{figure}

The plot shows the typical oscillatory pattern: when the prey population decreases, food becomes scarce for predators provoking a reduction in their numbers. As predator number falls, preys have more chances to grow causing a rise in their population. This causes a rebound in predator numbers, continuing the cycle with a time-shifted relation between the two groups.
